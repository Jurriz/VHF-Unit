\vspace{-10ex}%
\rule{\textwidth}{0.3pt}
\vspace{5ex}


The prototype could have been made directly using a company to place all the components, or the components should have been mounted with a pic and place machine to eliminate some initial troubleshooting and hassle regarding the SMT process. But with one of the project's goals to keep down the process looked like this. This did, however, increase the understanding and personal expertise in the area. 
\newline

The test points were a good addition, they were frequently used and employees at the company did find them interesting for further use in future projects.
Additional test points would be a good idea. When the board already had real estate to hold some additional points it should have been implemented. 

The hardest bit with the system was the voltage adjusting buck converter. Both the fact that it was one of the smallest pieces which made it hard to solder. Also the fact that it needed some adjustments in the form of 
\newline

Designing a PCB with white solder mask looks great and with black silkscreen, it was easy to see every marking. Problems did occur though, when visually inspecting the traces on the PCB it was hard almost impossible to see. There is a reason that green is the most common choice. (The final design was therefore created with green solder mask)

To ease troubleshooting more a via could be placed in every single transmission line, this to be able to read the voltage levels and data at every trace. When building the first revision only some of the traces.