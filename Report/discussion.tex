The prototype could have been made directly using a company to place all the components, or the components should have been mounted with a pic and place machine to eliminate some initial troubleshooting and hassle regarding the SMT process. But with one of the projects goals to keep to expences down the process looked like this. This did however increase the understanding and personal expertice in the area. 

The test points were a good addition, they was frequently used and eployes at the company did found them intresting for furter use in future projects.
Additional test points whould be a good idea. When the board already had real estate to hold some additional points it should have been inplemented. 

The hardest bit with the system were the voltage adjusting buck converter. Both the fact that it was one of the smallest pieces and also the fact that it needed some adjustments in form of 

Designing a PCB with white soldermask looks great and with black silcskreen it was easy to see every marking. Problems did occur though, when visually inspecting the traces on the PCB it was hard almost inpossible to see. There is a reason that green is the most common choise. (The final design was therefore created with green soldermask)

To ease trobleshooting more a via could be placed in every single transmission line, this to be able to read the voltage levels and data at every trace. When building the first revision only some of the traces.