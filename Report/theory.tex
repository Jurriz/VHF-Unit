This chapter will introduce the reader to the problem formulation and the underlaying principles taken into account. The priciple of radio communication and the important speps involved in developing a new product prototype will be presented.

\subsection{
Much considerations is taken in orderto construct the final product. The project includes both the software implementation and also the hardware design. As the constrution of the board 
This chapter will describe the function of each element and their role in the complete project. 
This chapter 



\subsection{Prototyping}
The final design is always the goal. In this case the final prototype shoud be as small and inexpensive as possible. Some of the design elements of this prototype is directly related to this. Some basic criteriums for manufacturing is taken from the PCB suppliers website. These parameters is some industry standards and also from the manufacturers side the limitations for their equipment. Whitout going up in cost but keeping the cost down with no extra price these are the.



\subsection{PCB}
To get good characteristics of the circuit the system will be produced on a four-layer PCB. This have a lot of advatages over a standard two layer board. The four layer design will have a top layer where all the components will be placed, a bottom layer with extra room for tracing tracks and having another ground pylogon. The two inner layers consist of a power plane and a ground layer. Both for easier connecability and also as a good way of making the circuit more resistant to high frequency noise between the traces on the top and bottom layer.



\subsection{Radio}


\subsection{MicroControllerUnit}
Every advanced system needs a central processing unit to make all the calculations needed. Why a lot of systems needs some powerful processor this system wich is a bit easier with 

To get the right speed of a crystal we need to get the right product for this particular case. As a crystal 
Many different types of oscillators exists, both internal and external types. 

\subsection{Accelerometer}