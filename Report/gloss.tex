
%%% ADC
%%% The glossary entry the acronym links to   
\newglossaryentry{adcg}{name={ADC},
    description={An Analog-to-digital converter (ADC) is a system that converts 
    an analog signal into a digital signal.}}

%%% define the acronym and use the see= option
\newglossaryentry{adc}{type=\acronymtype, name={ADC}, 
description={An Analog-to-digital converter (ADC) is a system that converts 
    an analog signal into a digital signal}, 
first={Analog-to-digital converter (ADC)\glsadd{adcg}}}


%%% CAD
\newglossaryentry{cadg}{name={CAD},
    description={Computer-aided design (CAD) is a computer system which aids
    the creation and modification of some kind of design.}}

\newglossaryentry{cad}{type=\acronymtype, name={CAD}, 
description={Computer-aided design (CAD) is a computer system which aids
    the creation and modification of some kind of design.}, 
first={Computer-aided design (CAD)\glsadd{cadg}}}


%%% DRC
\newglossaryentry{drcg}{name={DRC},
    description={Design Rule Check (DRC) is an automated check of your designed \gls{pcb}
	to see if it follows all specified design rules.}}
\newglossaryentry{drc}{type=\acronymtype, name={DRC}, 
description={Design Rule Check (DRC) is an automated check of your designed \gls{pcb}
	to see if it follows all specified design rules.}, 
first={Design Rule Check (DRC)\glsadd{drcg}}}


%%% ERC
\newglossaryentry{ercg}{name={ERC},
    description={Electronic Rule Check (ERC) is an automated check of your designed schematic
	to see if it follows all electronic rules.}}
\newglossaryentry{erc}{type=\acronymtype, name={ERC}, 
description={Electronic Rule Check (ERC) is an automated check of your designed schematic
	to see if it follows all electronic rules.}, 
first={Electronic Rule Check (ERC)\glsadd{ercg}}}


%%% GPIO
\newglossaryentry{gpiog}{name={GPIO},
    description={General Purpose Input Output (GPIO), is a general use port of an \gls{mcu}.}}

\newglossaryentry{gpio}{type=\acronymtype, name={GPIO}, 
description={General Purpose Input Output (GPIO), is a general use port of an \gls{mcu}.}, 
first={General Purpose Input Output (GPIO)\glsadd{gpiog}},}


%%% GPS
\newglossaryentry{gpsg}{name={GPS},
    description={The Global Positioning System (GPS) is a radionavigation
    system owned by the United States government and operated by the 
    United States Air Force. It uses sattelites for geolocation and time.}}

\newglossaryentry{gps}{type=\acronymtype, name={GPS}, 
description={The Global Positioning System (GPS) is a radionavigation
    system owned by the United States government and operated by the 
    United States Air Force. It uses sattelites for geolocation and time.}, 
first={Global Positioning System (GPS)\glsadd{gpsg}}}


%%% I2C
\newglossaryentry{i2cg}{name={I$^2$C},
    description={Inter-Integrated Circuit (I$^2$C), is a serial computer bus.}}

\newglossaryentry{i2c}{type=\acronymtype, name={I$^2$C}, 
description={Inter-Integrated Circuit (I$^2$C), is a serial computer bus.}, 
first={Inter-Integrated Circuit (I$^2$C)\glsadd{i2cg}}}


%%% IC
\newglossaryentry{icg}{name={IC},
    description={Integrated Circuit (IC) is a set of electronic circuits on one small flat chip.}}
\newglossaryentry{ic}{type=\acronymtype, name={IC}, 
description={Integrated Circuit (IC) is a set of electronic circuits on one small flat chip.}, 
first={Integrated Circuit (IC)\glsadd{icg}}}


%%% IPC
\newglossaryentry{ipcg}{name={IPC},
    description={Institute for Printed Circuits (IPC), Institute for Interconnecting and Packaging Electronic Circuits}}

\newglossaryentry{ipc}{type=\acronymtype, name={IPC}, 
description={Institute for Printed Circuits (IPC), Institute for Interconnecting and Packaging Electronic Circuits}, 
first={Institute for Printed Circuits (IPC)\glsadd{ipcg}}}


%%% IMU
\newglossaryentry{imug}{name={IMU},
    description={Inertial Measurement Units (IMUs) are integrated circuits that
    can measure acceleration, rotational velocity and magnetic field strength.}}

\newglossaryentry{imu}{type=\acronymtype, name={IMU}, 
description={Inertial Measurement Units (IMUs) are integrated circuits that
    can measure acceleration, rotational velocity and magnetic field strength.}, 
first={Inertial Measurement Unit (IMU)\glsadd{imug}}}


%%% LED
\newglossaryentry{ledg}{name={LED},
    description={A Light-emitting diode (LED), is a two-lead semiconductor light source.}}

\newglossaryentry{led}{type=\acronymtype, name={LED}, 
description={A Light-emitting diode (LED), is a two-lead semiconductor light source.}, 
first={Light-emitting diode (LED)\glsadd{ledg}}}


%%% LDO
\newglossaryentry{ldog}{name={LDO},
    description={A low-dropout regulator is a DC linear voltage regulator that can regulate the output voltage even when the supply voltage is very close to the output voltage.}}

\newglossaryentry{ldo}{type=\acronymtype, name={LDO}, 
description={A low-dropout regulator is a DC linear voltage regulator that can regulate the output voltage even when the supply voltage is very close to the output voltage.}, 
first={Low-dropout regulator (LDO)\glsadd{ldog}}}


%%% LGA
\newglossaryentry{lgag}{name={LGA},
    description={Land Grid Array (LGA) is a type of \gls{smd} device with pins on the socket.}}

\newglossaryentry{lga}{type=\acronymtype, name={LGA}, 
description={Land Grid Array (LGA) is a type of \gls{smd} device with pins on the socket.}, 
first={Land Grid Array (LGA)\glsadd{lgag}}}


%%% LI-ION
\newglossaryentry{liong}{name={LIB},
    description={Lithium-ion Battery (LIB) is a common type of rechargeable battery.}}

\newglossaryentry{lion}{type=\acronymtype, name={LIB}, 
description={Battery Management System}, 
first={Lithium-ion Batteries (LIB)\glsadd{liong}}}


%%% JTAG
\newglossaryentry{jtagg}{name={JTAG},
    description={JTAG is an industry standard for verifying designs and testing printed circuit boards after manufacture.}}

\newglossaryentry{jtag}{type=\acronymtype, name={JTAG}, 
description={JTAG is an industry standard for verifying designs and testing printed circuit boards after manufacture.}, 
first={Joint Test Action Group (JTAG)\glsadd{jtagg}}}


%%% MCU
\newglossaryentry{mcug}{name={MCU},
    description={A Microcontroller Unit (MCU) is a single computer chip 
    designed for embedded applications.}}

\newglossaryentry{mcu}{type=\acronymtype, name={MCU}, 
description={A Microcontroller Unit (MCU) is a single computer chip 
    designed for embedded applications.}, 
first={Microcontroller Unit (MCU)\glsadd{mcug}}}


%%% PC
\newglossaryentry{pcg}{name={PC},
    description={Personal Computer (PC)  is a multi-purpose computer whose size,
	capabilities, and price make it feasible for individual use.}}

\newglossaryentry{pc}{type=\acronymtype, name={PC}, 
description={Personal Computer (PC)  is a multi-purpose computer whose size,
	capabilities, and price make it feasible for individual use.}, 
first={Personal Computer (PC)\glsadd{pcg}}}


%%% PCB
\newglossaryentry{pcbg}{name={PCB},
    description={A Printed Circuit Board (PCB) is the common acronym when
	referring to populated circuit boards.}}
\newglossaryentry{pcb}{type=\acronymtype, name={PCB}, 
description={A Printed Circuit Board (PCB) is the common acronym when
	referring to populated circuit boards.}, 
first={Printed Circuit Board (PCB)\glsadd{pcbg}}}


%%% PTC 
\newglossaryentry{ptcg}{name={PTC},
	description={Positive Temperature Coefficient (PTC) describes the
	relative change of a physical property that is associated with a given change in temperature.}}
\newglossaryentry{ptc}{type=\acronymtype, name={PTC},
description={Positive Temperature Coefficient (PTC) describes the
	relative change of a physical property that is associated with a given change in temperature.},
first={Positive Temperature Coefficient (PTC)\glsadd{ptcg}}}


%%% PWB
\newglossaryentry{pwbg}{name={PWB},
    description={A Printed Wire Board (PWB) is the common acronym when
	referring to unpopulated circuit boards.}}

\newglossaryentry{pwb}{type=\acronymtype, name={PWB}, 
description={A Printed Wire Board (PWB) is the common acronym when
	referring to unpopulated circuit boards.}, 
first={Printed Wire Board (PWB)\glsadd{pwbg}}}


%%% RTC
\newglossaryentry{rtcg}{name={RTC},
    description={A Real time clock is an IC that is used to mesure time even when the main device is off.}}

\newglossaryentry{rtc}{type=\acronymtype, name={RTC}, 
description={A Real time clock is an IC that is used to mesure time even when the main device is off.}, 
first={Real time clock (RTC)\glsadd{rtcg}}, see=[Glossary:]{rtcg}}


%%% TI
\newglossaryentry{tig}{name={TI},
    description={Texas Instruments Inc. (TI) is an American technology company
	that designs and manufactures semiconductors and various integrated circuits. }}

\newglossaryentry{ti}{type=\acronymtype, name={TI}, 
description={Texas Instruments Inc. (TI) is an American technology company
	that designs and manufactures semiconductors and various integrated circuits.}, 
first={Texas Instruments Inc. (TI)\glsadd{tig}}}


%%% SEK
\newglossaryentry{sekg}{name={SEK},
    description={Swedish Krona (SEK) is the currency in Sweden.}}
\newglossaryentry{sek}{type=\acronymtype, name={SEK}, 
description={Swedish Krona}, 
first={Swedish Krona (SEK)\glsadd{sekg}}}


%%% SMD
\newglossaryentry{smdg}{name={SMD},
    description={Surface Mount Devices (SMD) are electric components soldered directly on a \gls{pcb} instead of using through-holes.}}

\newglossaryentry{smd}{type=\acronymtype, name={SMD}, 
description={Surface Mount Device}, 
first={Surface Mount Device (SMD)\glsadd{smdg}}}


%%% SPI
\newglossaryentry{spig}{name={SPI},
    description={Serial Peripheral Interface Bus (SPI), is a synchronous 
    serial communication interface specification used for short distance 
    communication, primarily in embedded systems.}}

\newglossaryentry{spi}{type=\acronymtype, name={SPI}, 
description={Serial Peripheral Interface}, 
first={Serial Peripheral Interface (SPI)\glsadd{spig}}}

%%% ST
\newglossaryentry{stg}{name={ST},
    description={STMicroelectronics (ST) is a French-Italian multinational 
    electronics and semiconductor manufacturer.}}

\newglossaryentry{st}{type=\acronymtype, name={ST}, 
description={STMicroelectronics}, 
first={STMicroelectronics (ST)\glsadd{stg}}}


%%% SWD
\newglossaryentry{swdg}{name={SWD},
    description={Serial Wire Debug (SWD) is an alternative 2-pin electrical interface,
	standard debugging protocol used in ARM processors.}}
\newglossaryentry{swd}{type=\acronymtype, name={SWD},
description={Serial Wire Debug}, 
first={Serial Wire Debug (SWD)\glsadd{swdg}}}


%%% UART
\newglossaryentry{uartg}{name={UART},
    description={Universal asynchronous receiver-transmitter (UART) is a computer hardware device for asynchronous serial communication.}}

\newglossaryentry{uart}{type=\acronymtype, name={UART}, 
description={Universal asynchronous receiver-transmitter}, 
first={Universal asynchronous receiver-transmitter (UART)\glsadd{uartg}}}


%%% USB
\newglossaryentry{usbg}{name={USB},
    description={Universal Serial Bus (USB), is an industry standard for
    cables, connectors and communications protocols for connection, 
    communication, and power supply between computers and devices.}}
\newglossaryentry{usb}{type=\acronymtype, name={USB}, 
description={Universal Serial Bus}, 
first={Universal Serial Bus (USB)\glsadd{usbg}}}


%%% VHF
\newglossaryentry{vhfg}{name={VHF},
    description={VHF}}
\newglossaryentry{vhf}{type=\acronymtype, name={VHF}, 
description={Portable Network Graphics}, 
first={Portable Network Graphics (PNG)\glsadd{vhfg}}}


%%% VIA
\newglossaryentry{viag}{name={VIA},
    description={A vertical interconnect access (via) is an electrical connection between
	layers in a physical electronic circuit.}}

\newglossaryentry{via}{type=\acronymtype, name={VIA}, 
description={vertical interconnect access}, 
first={Vertical Interconnect Access\glsadd{viag}}}


\thispagestyle{plain}